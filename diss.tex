\section{Introduction}

\subsection{Ypnos}

\emph{Stencil computations} are an idiom used in parallel programming.
They work by defining a \emph{kernel} (or \emph{stencil}) which is
applied to each element of an array of data (which I will call a
\emph{grid}). The kernel computes a new value for the array location
using the old value and the old values of its neighboring cells. In
summary the opperation behaves similarly to convolution.

The idiom is particularly useful in the fields of graphical processing
where some typical applications include: Gaussian blur, Laplacian of
Gaussian, Canny edge detection and many other filter based methods.
Stencils can also be used to computer approximations of differential
systems of equations. As such they are useful in the simulation of
physical systems via fluid, stress or heat dynamics.

Ypnos is an \emph{embedded domain specific language}(EDSL) for stencil
computations. Rather than build a language from scratch, it is embedded
within the Haskell programming language. This allows Ypnos to share much
of the syntax and implementation of its host language. Haskell is a
particularly good fit for stencil computations as its purity allows the
programmer to write parallel programs without worrying about the
interaction and sharing of state.

\subsubsection{Syntax}

The Ypnos language provides a custom syntax for defining stencil
functions as well as a collection of primative operations for their
manipulation and use.

The syntax for a simple averaging stencil would look as follows:

\begin{verbatim}
avg2D = [fun| X*Y:|_ a _| |b @c d| |_ e _| -> (a + b + c + d + e )/5|]
\end{verbatim}

Ypnos uses the \emph{quasiquoting mechanism} in Haskell to provide its
syntax. It essentially allows the programmer to provide a parser and
enter the custom syntax in brackets
\texttt{{[}parser\textbar{} ... my custom syntaxt ...\textbar{}{]}}. In
the case of the Ypnos stencil functions the parser is called
\texttt{fun}.

The next thing to be noted about the stencil is the syntax
\texttt{X*Y:}. \texttt{X} and \texttt{Y} are both dimension variables
(as is \texttt{Z}). They can be combined using the \texttt{*} operator.
The syntax defines the dimensionality of the stencil and helps us parse
the arguments.

The arguments are enclosed within pipe characters (\texttt{\textbar{}}).
Their arrangement in code is typically indented to reflect their grid
shape. Arguments can either be named or ``don't care'' possitions
denoted with either a name or an underscore respectively. The arguments
annotated with \texttt{@} is the cursor, the central cell whose position
is used for the result of the stencil.

The final section of the syntax comes after \texttt{-\textgreater{}} and
is the computation. This can be most Haskell syntax though recursion and
function definition is not possible.

\subsubsection{Primitives}

As well as the syntax for stencil functions, Ypnos provides a library of
primative operations. The primatives allow the programmer to combine the
stencils with grids to produce the computations they want. The main
primative in Ypnos is the \emph{run} primative which applies the stencil
computation to a grid.

\begin{verbatim}
run :: (Grid D a -> b) -> Grid D a -> Grid D b
\end{verbatim}

The application is done by moving the stencil cursor over each location
in the grid. The arguments of the stencil are taken from positions in
the grid relative to the cursor. The value is then computed using the
specified computation and put into the same cursor location in a
\emph{new} grid.

pppTODO: Say something about the argument being Grid D a.

In some locations near the edge of the grid their may not be enough
neighbors to satisfy a stencil. In this case Ypnos provides a special
syntax for dealing with these \emph{boundaries}. I have considered the
implementation of boundaries beyond the scope of this project however I
will include a brief description of their behaviour.

For each boundary of the grid outside of which the run primative may
need to access we define a behaviour. We may compute a value for these
locations using: the current location index, values from the grid
(accessed via a specially bound variable). A common boundary is the
\emph{mirror} boundary which works by providing the closest value inside
the grid when an outside access is made. This is the boundary that I
have tacitly assumed in my implementation.

Another vital primative of the Ypnos language is the \emph{reduce}
primative whose purpose is to summarise the contents of a grid in one
value. It may be used to compute functions such as the mean, sum or
minimum/maximum.

\begin{verbatim}
reduce :: (a -> a -> a) -> Grid D a -> a
\end{verbatim}

The primative uses an associative operator (of type
\texttt{a -\textgreater{} a -\textgreater{} a}) to combine all the
elements of the grid to one value. A more general version of this
operator also exists which support an intermediary type.

\begin{verbatim}
reduceR :: Reducer a b -> Grid D a -> a mkReducer :: exists b. (a -> b -> b)
-> (b -> b -> b) -> b -> (b -> c) -> Reducer a
\end{verbatim}

The \emph{Reducer} data type takes parameters:

\begin{itemize}
\itemsep1pt\parskip0pt\parsep0pt
\item
  a function reducing an element and a partial value to a partial value,
\item
  a function reducing two partial values,
\item
  a default partial value
\item
  and a conversion from a partial value to the final value.
\end{itemize}

This the Reducer is passed into the \emph{reduceR} primitive taking the
place of our associative operator in the reduce primitive. Clearly,
reduce can be implemented in terms of reduceR and so the later is the
more general.

\section{Preparation}

TODO: Summarise section

TODO: Make more discursive, story like.

TODO: Why am I doing all this?

\subsection{Accelerate}

Modern GPUs provide vast amounts of SIMD parallelism via general purpose
interfaces (GPGPU). For most users these are hard to use as the
interfaces are very low level and require the user to put a lot of
effort into writing correct parallel programs.

Matrices are a mathematical model which embodies SIMD parallelism. They
are operators over large amounts of data. It is possible to express many
parallel calculations and operations as matrix equations.

\emph{Accelerate} is an EDSL for the Haskell language which implements
parallel array computations on the GPU. The primary target GPUs are
those which support NVIDIA's CUDA extension for GPGPU programming.
Accelerate uses algorithm skeletons for online CUDA code generation.

Accelerate uses the Haskell type system to differentiate between arrays
on the CPU and GPU. It does this by way of a type encapsulating GPU
operations. There is a further \emph{stratification} of this type into
scalar and array values with scalar computations being composed into
array computations.

Accelerate contains built in support for stencil computations and so is
an excellent intermediary target for this project. While the stencil
semantics of Accelerate and Ypnos differ in some respects, the former is
powerful enough to represent the later. As such, this project will
concern itself not with the generation of CUDA code directly but through
the Accelerate language.

\subsubsection{GPU computation}

Haskell execution happens on the CPU and in main memory whereas GPU
execution happens in parallel in a separate memory. In order for a
process on the CPU to execute a CUDA program it must first send the
program and the data to the GPU. When the result is ready it must be
copied back into the main memory of the process concerned. I will call
these two processes \emph{copy-on} and \emph{copy-off} respectively.

Accelerate chose to represent this difference in the type system. The
\texttt{Acc} type denotes an operation on the GPU. For the purposes of
Accelerate, the only operations allowed on the GPU are those over
arrays. As such, \texttt{Array sh e} denotes an array of shape
\texttt{sh} and element type \texttt{e} and \texttt{Acc (Array sh e)}
denotes the same but in GPU memory and may also encapsulate an
operation.

Arrays are signalled for use on the GPU via the \texttt{use} primitive.
They are copied-on, executed and copied-off via the \texttt{run}. This
primitive is responsible for the run-time compilation and actual data
transfer. All other operations build an AST to be compiled by the run
primitive. Together use and run form the constructors and destructors of
the \texttt{Acc} data type.

\begin{verbatim}
use :: Array sh e -> Acc (Array sh e)
run :: Acc (Array sh e) -> Array sh e
\end{verbatim}

\subsubsection{A stratified language}

While the main type of operation in Accelerate is over arrays. However,
we often want to compose arrays out of multiple scalar values or
functions over scalars. A classic example of this is the map function to
transform an entire array by a function over the individual
values\footnote{In fact, the map function is conceptually similar to
  stencil application. The difference being that stencils also take into
  account the neighbourhood of a cell to compute the next value.}. For
this reason, in addition to the \texttt{Acc} type, Accelerate also
provides the \texttt{Exp} type where the former represents collective
operations and the later represents scalar computations.

The map function would then looks like this:

\begin{verbatim}
map :: (Exp a -> Exp b) -> Acc (Array sh a) -> Acc (Array sh b)
\end{verbatim}

Scalar operations do not support any type of iteration or recursion in
order to prevent the asymmetric operation run time. However, most other
Haskell code is allowed. This is achieved by the Haskell class
mechanism: Accelerate provides instances of \texttt{Exp a} for most
common classes.

For example, in the support of addition, subtraction and other numerical
operations, Accelerate provides an instance of the type class
\texttt{Num}. This means that operations can be typed as follows:

\begin{verbatim}
(+) :: Num a => Exp a -> Exp a -> Exp a
1 + 2 + 3 :: Exp Integer
\end{verbatim}

\subsubsection{Stencil support}

I have already mentioned that function over scalars can be applied over
a whole grid, the map function being an example of this. Accelerate
provides support for stencil computations via the \texttt{stencil}
function.

\begin{verbatim}
stencil :: Stencil sh a sten =>
           (sten -> Exp b) ->
           Boundary a ->
           Acc (Array sh a) ->
           Acc (Array sh b)

instance Stencil DIM2 a ((Exp a, Exp a, Exp a)
                        ,(Exp a, Exp a, Exp a)
                        ,(Exp a, Exp a, Exp a))
\end{verbatim}

The first parameter is a function which represent the stencil. We see
that \texttt{sten}, the stencil pattern, takes the form of a tuple grid
of \texttt{Exp a} element type. This allows Accelerate to use Haskell's
function syntax to define stencils.

The second parameter is the type of boundary. In Accelerate, the types
of boundary allowed are fixed as opposed to Ypnos boundaries which can
be fully specified. One of the types allowed is \texttt{Mirror} which
deal with an out of bounds access by picking the nearest coordinate from
within the array.

With these two parameters we have defined an operation performs the
stencil convolution.

\subsection{Learning the libraries}

In order to get familiar with the syntax of these two libraries I
decided to implement some sample functions in both, the main one being
the average function we have already seen. With this I was able to
familiarise myself with the embedded languages and prepare for the task
of implementation.

Furthermore, this allowed me to become familiar with the other tools I
would be using in this project, namely:

\begin{itemize}
\itemsep1pt\parskip0pt\parsep0pt
\item
  Haskell, the programming language. Having not used this before I had
  to get familiar with the syntax as well as some of the more advanced
  type system features and extensions such as: \emph{type classes},
  \emph{type families} and \emph{data families}.
\item
  Cabal, a build system for Haskell with automatic dependencies
  resolution and fetching. The toy functions allowed me to test setting
  up a build system which I would later use for the rest of the project.
\item
  Git, the version control system. I was already quite familiar with
  this system before this project but it allowed me to make use of some
  of Git's more advanced features such as \emph{stashing},
  \emph{sub-projects} and \emph{branching}.
\item
  CUDA, the drivers and library. As I do not own a machine with a CUDA
  enabled graphics card, I was using a remote machine located in the
  Computer Laboratory. The sample functions allowed me to set up the
  machine with the drivers and configuration required in order to run
  the Accelerate library.
\end{itemize}

TODO: Requirements analysis plus other software engineering techniques

\subsection{Test driven development}

The correctness of my implementation was a central goal from the
beginning of the project. In order to achieve this I took a test driven
approach to development. This meant that while writing the
implementation I was simultaneously writing unit tests for that code.
The approach allowed me to quickly and effectively find bugs which had
not already been found by the Haskell type system.

\emph{QuickCheck} is Haskell's defacto standard unit testing library. In
most unit testing libraries for other platforms, the programmer has to
provide sets of test data for the library to check against the program.
The code for generating this data is left to the programmer. QuickCheck
takes a different approach. Instead of specifying testing functions
which include the test generation, we specify properties which take the
data to be tested as an argument. We then leave the generation of this
data up to the library.

QuickCheck is able to generate random testing data for most built in
Haskell data types. For user defined types, the programmer must provide
an instance of the class \texttt{Arbitrary} which allows QuickCheck to
generate random samples for testing.

\section{Implementation}

\subsection{Compilation of the stencils}

Compilation of stencils was a central task in this project. The abstract
Ypnos syntax allows much flexibility in the underlying implementation.
Ypnos achieves this via Haskell's Quasi Quoting mechanism for compiling
custom syntax to Haskell AST. Accelerate's implementation has overridden
much of the Haskell operators required for this translation stage so the
bulk of the effort went into producing the functions that contained the
computation. These functions take the following form:

\begin{verbatim}
avg :: Exp a => Stencil3x3 a -> Exp a
avg (( _, a, _ )
    ,( b, c, d )
    ,( _, e, _ )) = (a + b + c + d + e) / 5

type Stencil3x3 = ((Exp a, Exp a, Exp a)
                  ,(Exp a, Exp a, Exp a)
                  ,(Exp a, Exp a, Exp a))
\end{verbatim}

The arguments are formed as tuples of tuples. The rest of the stencil
appears to be normal Haskell code. However, the return type,
\texttt{Exp a}, insures that all the operations actually use Accelerates
overridden methods to build an AST. The AST is then translated at
run-time into CUDA code.

Haskell's quasiquoting mechanism is a compiler option which allows the
library author to provide custom syntax for domain specific languages.
As such, it is a perfect fit for Ypnos, which would like to hide the
underling implementation from the user. It works by providing a parser
object (refered to as a quasiquoter) and a syntax for applying it within
normal Haskell code. The essential function of a quasiquoter is to
provide an abbreviation for entering the AST manually.

Take for example the situation in which we want to write an embedded
language to act as a calculator. We have the following AST for our
simple calculator:

\begin{verbatim}
data Expr  =  IntExpr Integer
           |  BinopExpr (Integer -> Integer -> Integer) Expr Expr

e1 = BinopExpr (+) (IntExpr 1) (IntExpr 3)
e2 = [expr| 1 + 3 |]
\end{verbatim}

We see that the quasiquoter \texttt{expr} allows us to appreviate the
expression \texttt{e1} to the more obvious form of \texttt{e2}.

Clearly, if we were to swap out the quasiquoter this would be an
effective way of producing multiple programs from the same syntax. This
is what Ypnos achieves in its stencil syntax. The aim is to be able to
change the quasiquoter and fully change the underlying implementation
without any other modifications.

We could sensibly do the translation from Ypnos to Accelerate stencils
in one of two ways: we (a) use Haskell's type system to mask the
difference between the two types of stencil computation or (b) we use
run-time conversion to mask the difference between the implementations
and maintain the semblance of the types. I explored each of these
approaches in the course of the project and I would like to now describe
the benefits and drawbacks of both.

\hyperdef{}{type-sys}{\subsubsection{A type system
approach}\label{type-sys}}

As we saw in the previous section: the types of the Ypnos CPU stencil
and the Accelerate library's stencil differ wildly. Let's a closer look
at the precise differences between them in the types of our stencil
\texttt{avg}\footnote{For the sake of simplicity I have excluded the
  type constraints relating to boundaries as these are very long and
  complicated.}:

\begin{verbatim}
avgCPU :: (IArray UArray a, Fractional a) =>
          Grid (Dim X :* Dim Y) b dyn a -> a
avgGPU :: Floating (Exp a) =>
          Stencil3x3 a -> Exp a
\end{verbatim}

In the GPU case we see that the type (once expanded) is tuples of tuples
of \texttt{Exp a}. This allows Accelerate to make use of the built in
Haskell syntax for functions. This is of little use to us as we have our
own syntax already. On the other hand, in the CPU case we see that
arguments take the form of a grid, which is exactly the same type as the
grids it operates on.

This is no accident as Ypnos grid type is a comonad, the theoretic dual
of the of the monad. This restrains the type of the run operation to be
of the form:

\begin{verbatim}
cobind :: (D a -> b) -> D a -> D b
\end{verbatim}

Where we let \texttt{D} be a grid of a certain dimension and \texttt{a}
and \texttt{b} be the types of that grid.

Unfortunately, by translating directly to the Accelerate stencil type we
lose the comonadic nature of the type. This is a shame because this type
is both informative to the programmer yet flexible enough that by
changing the instance of \texttt{D} we change the implementation.

The advantage of this method (as we will see more in detail when we
discuss the alternative) is that all the translation effort is done at
compile time allowing the running of the stencil to be more efficient.

Another way in which Accelerate and Ypnos stencils differ is that the
former assumes that the cursor is centred whereas the later allows the
user to program this. This can be translated by padding the stencil
handed off to Accelerate such that the cursor has been centred.

This is perhaps best illustrated by example. Say that we have the
following one dimensional stencil with the cursor at an off-centre
location (denoted by \texttt{c}).

\begin{verbatim}
          b
  *---------------*
  | _ | c | _ | _ |
  *---*   *-------*
    a       b-a-1
\end{verbatim}

We define the variables $a$ to be the position of the cursor and $b$ to
be the length of the stencil. Now we must find how much we must pad at
the beginning and end of the stencil to centre the cursor. This is given
by the following two equations:

\[ pad_{beginning} = max \{a, b\} - a \]

\[ pad_{end} = max \{a, b\} - b + a + 1 \]

This means that after centring we get the following:

\begin{verbatim}
         coffset roffset
          *---*   *---*
  | _ | _ | c | _ | _ |
\end{verbatim}

\subsubsection{Implementing centring}

In order to implement the centring I had to consider both the one and
two dimensional cases. It would be quite easy to deal with this deal
with this in two separate cases except that it would be nice to extend
the approach to higher dimension eventually. I three principle
approaches to doing this: using lists as an intermediary; using arrays
as intermediaries; and operating on the grid patterns directly via type
classes. Before addressing the approaches I will mention the types we
were converting.

\begin{verbatim}
data GridPattern =
    GridPattern1D DimTag [VarP] |
    GridPattern2D DimTag DimTag [[VarP]]
\end{verbatim}

\texttt{GridPattern} is the type in the Ypnos AST corresponding to the
parsed pattern of arguments. We see that it takes both a 1D and 2D form
where the variables (\texttt{VarP}) are a list and a list of lists
respectively. We may also note that the dimensionality is expressed
directly in the constructor and as such is not present in the type.

The pattern of arguments in Accelerate is a expressed as tuple in the 1D
case and a tuple of tuples in the 2D case. This representation contains
no information about which variables are cursors as we discussed in the
previous section.

\paragraph{Intermediate approaches}

The first approach taken involved converting first from grid patterns
into lists, balancing these lists then converting them into the centred
tuples needed for the Accelerate functional representation. In order to
do this I would have to define functions for measuring the location of
the cursor, and padding the lists before and after. This approach proved
difficult as lists did not explicitly incorporate their dimensionality
in their type. This made it hard to treat the 1D and 2D cases
differently.

The second approach attempted to use existing array code in order avoid
writing such functions. The hope was that by converting to arrays,
rather than lists, functions for appending and prepending rows and
columns would already exist. However, this was not the case and I would
have had to write these myself. This made the point of the intermediary
stage of arrays altogether pointless.

\paragraph{Direct approach}

The third and final approach was to operate directly on the lists
extracted from the \texttt{GridPattern} types. As I already mentioned:
the problem with working with lists is that the dimensionality is lost
in the type. To retain this information in the type system I designed a
class \texttt{GridIx} to perform the basic operations -
\texttt{addBefore}, \texttt{addAfter}, \texttt{find} and \texttt{size} -
in a dimension sensitive way while still being polymorphic.

\begin{verbatim}
class (Ix i, Num i, ElMax i) => GridIx i where
    data GridPatt i :: * -> *
    addBefore :: i -> a -> GridPatt i a -> GridPatt i a
    addAfter :: i -> a -> GridPatt i a -> GridPatt i a
    find :: (a -> Bool) -> GridPatt i a -> i
    size :: GridPatt i a -> i
\end{verbatim}

The associated data type \texttt{GridPatt} would take the type of the
particular dimensionality of list that is appropriate for a given
instance. In the case of the index type \texttt{Int} we would get
\texttt{GridPatt Int a = {[}a{]}} and in the case of \texttt{(Int, Int)}
we get \texttt{{[}{[}a{]}{]}}. This approach allows the algorithms for
centring to be described at a general level without being concerned
about the number of dimensions actually involved.

\hyperdef{}{run-time}{\subsubsection{A run-time
approach}\label{run-time}}

The second approach to the translation of stencils was to keep the types
the same (or similar, as we will see) to Ypnos' original implementation.
This is alluring as it allows us to both expose more information to the
user through the programs type and maintain the theoretic underpinnings
of Ypnos: the comonadic structure. In order to achieve this we need to
do some run-time type translations. These have an overhead for (TODO: do
they?) the performance of the stencil application but I will discuss at
the end of this section what approaches can be taken to mitigate this
overhead.

As already seen, we would like the \texttt{run} primitive to take the
form:

\begin{verbatim}
run :: Comonad g => (g a -> b) -> g a -> g b
\end{verbatim}

We have also seen that Accelerate does not accept stencils of this form.
To solve this we previously broke the the comonadicity of the operation
but we could attempt preserve it by introducing an \emph{arrow} data
constructor to absorb the differences in type between Accelerates notion
of a function and Ypnos'. This changes the run function to:

run :: Comonad g =\textgreater{} (g a \texttt{arr} b) -\textgreater{} g
a -\textgreater{} g b

The data constructor is paramatrized on both \texttt{g a} and
\texttt{b}. To build up an instance of \texttt{arr} we must pass in the
stencil function to a special constructor. The constructor chosen
decides the implementation used.

While previously we had to use different versions of the quasiquoter to
produce different stencils at compile-time, we now use the same
quasiquoter but convert the function at run-time. We achieve this by
taking advantage of Haskell's polymorphism which allows a function over
type \texttt{a} to generalise to a function of type \texttt{Exp a}. This
generalisation in concert with the arrow data constructor allows our
stencil functions to have the type:

stencil :: Comonad g =\textgreater{} g (Exp a) -\textgreater{} Exp b
stencil' :: Comonad g =\textgreater{} g a \texttt{arr} b

Because of the arrow type, \texttt{stencil} and \texttt{stencil'} can
actually have the same type.

However, we are only half-way there: the type of stencil accepted by
Accelerate is still not of the form
\texttt{g (Exp a) -\textgreater{} Exp b}. I achieve this stencil by a
conversion function which builds an Accelerate stencil (call it
\emph{stencil A}) at run-time using the stencil encapsulated in the
arrow data type (call it \emph{stencil B}). Stencil A's arguments are
used to build up a grid of type \texttt{g (Exp a)} then stencil B is
used on this grid to produce the result of type \texttt{Exp b}.

While this run-time conversion creates an overhead it also, as we have
seen, simplifies the types significantly. However TODO: mention
deforestation.

\subsection{Implementation of the primitives}

TODO: rewrite

The primitives are the second central component of the translation.
Without them we could not run our translated stencils on the GPU. Like
the stencil translation the implementation of the primitives took two
primary approaches. The first was to re-implement the primitives in a
separate module. In this case the user would import whichever
implementation they required. This approach had some fatal draw backs in
that it required the user to change too much of their code between
implementations. This led to the second approach of extracting the
functionality of the primitive into a type class. This approach required
the use of some complicated type features in order to make the types
unify. However, the final result is much more usable.

\subsubsection{Non unifying approach}

The initial of run was linked to the compile time implementation of the
stencil function. At the highest level this meant that the function
\texttt{run} had the following type:

\begin{verbatim}
run :: (Stencil sh x sten) =>
       (sten -> Exp y) -> Grid d Nil Static x -> Grid d Nil Static y
\end{verbatim}

However, we see that the type of \texttt{sh} (required by Accelerate)
and \texttt{d} (required by Ypnos) do not unify directly requiring
another constraint to reconcile the two. Further, constraints need to
then be added for the types of \texttt{x} and \texttt{y} to satisfy
Accelerates \texttt{stencil} function. In the end the end this type
becomes unwieldy meaning that it is not straight forward for the user to
replace it in their code.

Similar problems would have plagued the implementation of the
\texttt{reduce} primitive. However, having seen the first implementation
of the \texttt{run} primitive I decided that a different approach was
need so this incarnation of the \texttt{reduce} primitive never saw the
light of day.

\subsubsection{Type classes}

In this project I am aiming both to make and accurate and fast
translation as well as one which is easy for the programmer to use.
Practically, this means that converting between CPU and GPU
implementations of the same program should require minimal code changes.
With the previous approach we saw this did not work for two reasons: (a)
the run primitive I implemented was not related (as far as Haskell was
concerned) to the original CPU primitive, and (b) the types of the two
primitives differed which could cause compilation to fail if they were
swapped.

What would be nice is to have one function which behaves differently
under certain program conditions. The perfect tool for this job is adhoc
polymorphism which is provided in Haskell via type classes. The result
is an implementation of the primitive which changes dependant on a
particular type parameter. The obvious parameter in our case is the grid
type as this is common to all Ypnos primitives and so can universally
(across all primitives) define whether to use a CPU, GPU or other
backend.

We have seen this before in some of the code examples I have used the
notation: ``Comonad g'' to refer to a grid which implements the
primitives of Ypnos. This is the same thing. However, we run into the
same problems as with stencil translation (see the section on
\hyperref[run-time]{run-time stencil translation}).

\subsubsection{Type class parameter}

The first approach to solving this problem makes use of the fact that
Haskell type classes can be parametrized on more than one type. This
allows us to extract parts of the type that change to give a
(semi-)unified type. As the reduce primitive was the first to bring
about such issues lets examine how this approach can be applied to it.

\begin{verbatim}
class ReduceGrid grid a b c | grid -> a,
                              grid -> b,
                              grid -> c where
    reduceG :: Reducer a b c-> grid -> c

data Reducer a b c where
    Reducer ::   (a -> b -> b)
              -> (b -> b -> b)
              -> b
              -> (b -> c)
              -> Reducer a b c

instance ReduceGrid CPUGrid a b c
instance ReduceGrid GPUGrid (Exp a) (Exp b) (Exp c)

class RunGrid grid sten | grid -> sten where
    runG :: sten -> grid -> grid

instance RunGrid CPUGrid CPUStencil
instance RunGrid GPUGrid GPUStencil
\end{verbatim}

In this approach we are able to have instances for \texttt{Reducer} for
the CPU and GPU based on the grid type yet we also change the types of
values accepted by the funtcions of the reducer. These values correspond
to different types of functions which is what tells Haskell to use the
Accelerate overloaded versions of operators.

We also see that the \texttt{RunGrid} type class is treated in a similar
manner: the type of grid uniquely determines the type of stencil
function required \footnote{The notation that denotes this in Haskell is
  \texttt{grid -\textgreater{} sten}. We see this a couple of times in
  the given example.}. Unlike the \texttt{reduceG} example, Haskell
cannot, without help from the programmer, choose a different quasiquoter
(as is required with the \hyperref[type-sys]{static approach}). With the
run-time approach we may be able to do better but we will see this
later. (TODO: ensure this forward reference holds)

So in theory this approach should work however, we encounter problems
with the useability of this approach. Let's further examine the the type
of the \texttt{reduceG} primitive when applied to \texttt{GPUGrid}s:

\begin{verbatim}
Reducer :: (Exp a -> Exp b -> Exp b)
        -> (Exp b -> Exp b -> Exp b)
        -> (Exp b) -- Default value
        -> (Exp b -> Exp c)
        -> Reducer (Exp a) (Exp b) (Exp c)
reduceG :: Reducer (Exp a) (Exp b) (Exp c)
        -> GPUGrid
        -> Exp c -- Return value
\end{verbatim}

Notice that both the return value and default value have type
\texttt{Exp} which is problematic as \emph{lifting} and \emph{unlifting}
is not easy for the user to do and the wrapped value is not particularly
useful or meaningful. One approach to changing this would be to
introduce dependant type parameters for the functions rather than the
values and this could work however, I actually took the following
approach.

\subsubsection{Associated type families}

We already encountered associated type families in the section on
{[}direct translation{]}\{\#direct\}. Here we will use these same type
families to achieve the different function types we are looking for.

The ideal type for the \texttt{Reducer} in the GPU implentation would
be:

\begin{verbatim}
Reducer :: (Exp a -> Exp b -> Exp b)
        -> (Exp b -> Exp b -> Exp b)
        -> b
        -> (Exp b -> Exp c)
reduceG :: Reducer a b c -> GPUGrid -> c
\end{verbatim}

Clearly this is an improvement to the user as they get a simple value
they know what to do with. By examining this we can deduce that there
are actually two of abstract function involved: 1 arguments functions of
\texttt{Exp}s and 2 argument. If we implement these we as two associated
type families we get the behaviour we want:

\begin{verbatim}
Reducer :: Fun2 g a b b
        -> Fun2 g b b b
        -> b
        -> Fun1 g b c

class ReduceGrid g where
    type Fun1 g a b
    type Fun2 g a b c
    reduceG :: Reducer g a b c -> g -> c
\end{verbatim}

Next I wanted to extend this approach to run. However, with the run
primitive we do not simply have a conversion of types but also
conditions on those types (called contexts in Haskell). It is possible
to encode contexts in a type family method using a Haskell language
extension called \emph{ConstraintKinds}. This allows us to define a type
family has the \emph{kind} of \texttt{Constraint} instead of the usual
\texttt{*} (denoting type). Here is an example of the \texttt{RunGrid}
class modified in this way:

\begin{verbatim}
class RunGrid g where
    type ConStencil g a b sten :: Constraint
    type Stencil g a b sten :: *
    run :: ConStencil g a b => (Stencil g a b) -> g x -> g y

instance RunGrid g where
    type ConStencil g a b sten = (Stencil sh a ~ sten, ShapeOf g ~ sh)
    type Stencil g a b sten = sten -> Exp b
\end{verbatim}

NEXT: exposed sten. Not very nice or general

\subsubsection{Associated data families}

\subsubsection{Final implementation}

TODO: run implementation

TODO: reduce implementation

\subsection{How would it be used}

TODO: Example of usage and replacement

\subsubsection{Constructors and destructors}

\section{Evaluation}

The main aims of this project were to produce a correct translation and
speed up over the CPU implementation. In order to test these two goals I
have implemented unit testing through out the course of this project and
implemented an evaluation suite of programs. The GPU is a type of
co-processor and as a result incurs an overhead for copying results to
and from its local memory. In evaluating the speed up of using the GPU I
have to account for this.

\subsection{Speed-up}

Before embarking on the evaluation I postulated that the GPU should
provide a speed up over the CPU due to its capacity for parallel
computation. Seeing as the stencil computation is highly data parallel
it is a perfect fit for the SIMD parallelism of the GPU. More
specifically, I expected that: as grid sizes increased the run time of
the computation would increase less quickly in the GPU case compared
with the CPU case.

\subsubsection{Methodology}

To measure the run-time I made use of a library called \emph{Criterion}
which provides functions for:

\begin{itemize}
\itemsep1pt\parskip0pt\parsep0pt
\item
  Estimating the cost of a single call to the \texttt{clock} function.
  Which does the timing of the CPU.
\item
  Estimating the clock resolution.
\item
  Running Haskell functions and timing them discounting the above
  variations in order to get a sample of data.
\item
  Analysing the sample using \emph{bootstrapping}{[}TODO: link to
  paper{]} to calculate the mean and confidence interval.
\end{itemize}

In my experimental setup I am using a confidence interval of 95\% and a
sample size of 100 and a resample size of 100,000. The result from
Criterion is a mean with a confidence interval of 95\%. I will use these
results to compare the performance of the various functions implemented.

The machine being used for benchmarking was provided by the labs and
remotely hosted. The machines specifications are as follows:

\begin{itemize}
\itemsep1pt\parskip0pt\parsep0pt
\item
  Ubuntu Linux 12.04 32-bit edition
\item
  Quad core Intel Core i5-2400S CPU clocked at 2.50GHz with a 6M cache
\item
  16GB of core memory
\item
  Nvidia GeForce 9600 GT graphics card featuring the G94 GPU with a 256M
  framebuffer.
\end{itemize}

\subsubsection{Overhead}

In order to show this I must first discount the effect of copying to and
from the GPU. This was done via an \texttt{id} function implemented in
Accelerate. The effect of which is to copy the data from the CPU to the
GPU, perform no operations there then copy the data back. This will
allow us to have a baseline measure of how fast our computations could
be without this overhead.

\subsubsection{Benchmark suite}

The benchmark suite must test both the speed-up of both primitives:
\texttt{run} and \texttt{reduce}. For this I have implemented a set of
functions representative functions for each to test speed across a
representative set of calculations. These functions include:

\begin{itemize}
\itemsep1pt\parskip0pt\parsep0pt
\item
  \textbf{The average stencil} {[}ref{]} that we have seen in the
  previous sections. This function is representative of convolution
  style operations which we may wish to perform on the data. It operates
  over floating point numbers which is a common use case for scientific
  computing
\item
  \textbf{The Game of Life stencil} makes use of various boolean
  functions as well as externally declared functions used to count the
  number of \emph{true} values in a list.
\item
  \textbf{The sum} and \textbf{mean reduction functions} which
  constitute two of the most common reduction operations over grids.
\end{itemize}

TODO: Speed up run

TODO: Reduce

\subsubsection{Deducing a model}

TODO: Model of on/off

\subsection{Unit testing}

A central goal of the project was to produce a correct translation from
Ypnos to Accelerate. Already, by choosing a type safe language such as
Haskell I vastly reduced the number of run-time errors possible due to
programming errors. To catch the rest I made use of \emph{unit testing}
and \emph{Test Driven Development}. Clearly, unit testing can only
provide an assurance of correctness and not a guarantee. However, I
decided that a formal proof (which could give these guarantees) was
beyond the scope of this project. In writing these test I have assumed
that the original CPU implementation was correct and could be compared
against as a gold standard.

The testing framework used works slightly differently to other unit
testing frameworks. In a standard framework the user provides test cases
which incorporates both the test data (some times generated) and
assertions. In Haskell's \emph{QuickCheck} we only provide axioms about
our functions and let the framework provide the data based on the type.

Typically QuickCheck will generate hundreds of test samples to verify a
particular axiom. This provides a better assurance than ordinary unit
testing as via the random process, QuickCheck often comes up with corner
cases the programmer may not have devised themselves.

The following sections of my project where particularly necessary to
check by QuickCheck:

\begin{itemize}
\itemsep1pt\parskip0pt\parsep0pt
\item
  The centring algorithm for grid patterns, as this contains a large
  part of the translations complexity.
\item
  The \texttt{run} primitive.
\item
  The \texttt{reduce} primitive.
\end{itemize}

The approach taken to testing the grid patterns was ensure that the
transformation:

\begin{itemize}
\itemsep1pt\parskip0pt\parsep0pt
\item
  Starts with a grid that has certain properties (a precondition):
  regular size, positive size, has a cursor.
\item
  Maintains the regularity of size: the length of each row was is same.
\item
  Centres the cursor given the original grid had a cursor.
\item
  Both roffset and coffset are always positive on such a grid. {[}TODO:
  section reference{]}
\end{itemize}

The assumption was that grid patterns given to the transformation
procedures would be correct to begin with. As such, to improve the
amount of test data generated, I enforced these properties at the
generation level. This is safe as the grid patterns are generated
through the CPU translation which I am assuming to be correct.

To test the primitives I used a standard testing approach of comparing
against a existing correct implementation. Both implementations are fed
the same data and their results should come out the same. For the
\texttt{reduce} primitive I compare against Haskell's built in reduce
function as I can safely assume this to be correct. For the \texttt{run}
primitive I originally indented to test against the Ypnos CPU
implementation as I was assuming this to be correct. However, in running
my tests I uncovered a bug in the implementation of boundaries\footnote{{[}TODO:
  explain bug{]}} which made me consider other options.

Given that I couldn't trust the results of the CPU implementation I
tested the GPU primitive against a hand coded stencil in Accelerate.
This was not ideal as it used essentially the same code an the run
implementation but this still provided some assurance. Once the bug had
been fixed in the CPU implementation I was then able to test against
this as well.

The run primitive is tested by running the average function on a
randomly generated grid. The grid is passed to the GPU, CPU and
Accelerate implementations of \texttt{avg} the resulting grid is then
compared between the two and any difference counts as a failure.

The same procedure is used for the reduce primitive. We use a
one-dimensional grid for this case as the built in Haskell function we
are comparing against is one-dimensional. The resulting reduced values
are compared and an failure is registered if they should differ.

\begin{verbatim}
write  --->  compile   --->   test   --->   commit
code             |              |              |
  ^--------------+--------------+--------------+
\end{verbatim}

For the large part of the project I have been coding tests and
implementation in parallel (also known as Test Driven Development or
TDD). This allowed me to catch errors early on and fix them immediately.
TDD allowed for much faster debugging as it provides confidence in the
functionality of certain parts of code. This meant that when I
encountered bugs I was able to pin-point their origin often without the
use of a debugger.

\subsubsection{Useability to the programmer}

While not mentioned in my original proposal, the useability to the
programmer is another non-functional requirement. I decided that
performing a full usability study would be unnecessary as this was a
secondary requirement. Instead I have chosen to evaluate the usability
using the method of \emph{Cognitive Dimensions}{[}TODO:ref{]} to compare
the various approaches already discussed.

Cognitive Dmensions of notations (CD) provide a light weight vocabulary
for discussing various factors of programming language design. As Ypnos
is essentially a programming language (albeit, one embedded in Haskell)
it makes sense to use this technique. It works by specifying a number of
properties of a notation (\emph{dimensions}, for a complete list of
dimensions considered see appendix TODO) which must, in general, be
traded off against one another. For this reason it is important to
understand the representative tasks and the user that will be performing
them. Then design decisions in the language can be compared and
evaluated using the dimensions relative to the tasks.

\paragraph{System = Language + Environment}

It is important to note that CD relates to a whole system not just the
language. We define the system to be the combination of programming
language and the programming environments. For example, programming over
the phone verses programming in a visual editor. For the purposes of
discussing only the language changes that I have introduced I will fix
environment and assume that it has the following features:

\begin{itemize}
\itemsep1pt\parskip0pt\parsep0pt
\item
  Screen-based text editor (e.g.~Vim, Emacs or TextMate)
\item
  Search and replace functionality (including regular expressions)
\item
  {[}TODO: Anything else?{]}
\end{itemize}

\paragraph{Methodology}

I used the following procedure in evaluating the changes to Ypnos using
CDs:

\begin{itemize}
\itemsep1pt\parskip0pt\parsep0pt
\item
  Identify the relevant users of my system and sketch out a basic user
  profile.
\item
  Select the relevant task of these users on my part of the language.
\item
  Highlight which cognitive dimensions are most important to tasks
  selected.
\item
  Show a comparison of the various approaches to this implementation.
\item
  Conclude which approach was taken and why.
\end{itemize}

\paragraph{User profiles}

I have decided that given the applications to scientific computing and
graphics the two main users of Ypnos would be scientists simulating
physical systems and graphics programmers developing graphics
algorithms. I have included two user stories for our two representative
users:

\begin{quote}
Kiaran is a physical scientist who is writing a simulation of a fluid
dynamics system. He has a little Haskell experience already but has
mostly used other languages such as Matlab and Fortran. He chose
Ypnos/Haskell because he knew it would allow him to easily switch
between a CPU implementation on his machine and a GPU implementation on
the simulation machine he is using.
\end{quote}

\begin{quote}
Noemi is a writing a graphics transformation for a photo editing
package. The photos her user edit are typically very large but she still
would like to provide real-time performance with her algorithms. Noemi
has a GPU in her computer so she will be writing for this to ensure that
her performance is good. However, she also wants her system to degrade
well on machines that don't have a compatible GPU. She already has very
good experience in Haskell and is familiar with more complex features
and extensions such as type and data families. She has picked
Ypnos/Haskell because of it's syntax and the ease of degrading.
\end{quote}

We can see that there are many tasks that these users would want to
perform with our system: coding up a filter into a stencil (Noemi),
writing a complex reduction to determine the state of the system
(Kiaran), debugging to find out why they get the wrong values (both).
However, I will be ignoring all task that involve parts of the system
which I did not implement. This leaves us with one central task for the
two use cases: converting between GPU and CPU.

The cognitive dimensions relevant to this task are:

\begin{itemize}
\itemsep1pt\parskip0pt\parsep0pt
\item
  Low repetition viscosity: to allow the user to easily change the
  implementation without changing too many points in code.
\item
  Little to no imposed look ahead: allowing the programmer to use one
  implementation without having to think about later switching.
\item
  Consistency: the programs syntax or usage doesn't change from CPU to
  GPU.
\item
  Terseness: the syntax to specify the implementation doesn't get in the
  way of coding the stencils.
\item
  Closeness of mapping: the model presented to the user through the API
  should map well to the users mental model for these types of
  operation.
\end{itemize}

The various approaches to provide an API to the programmer where
discussed in the implementation section (TODO: link). They essentially
boiled down to the following three approaches: choosing the different
implementation based on importing, using type classes with associated
data families, using type classes with associated type families. For the
sake of comparison I will also include the approach of the programmer
re-coding their implementation in Accelerate for the GPU.

\begin{longtable}[c]{lllll}
\hline\noalign{\medskip}
\begin{minipage}[t]{0.14\columnwidth}\raggedright
Cognitive dimensions
\end{minipage} & \begin{minipage}[t]{0.21\columnwidth}\raggedright
Accelerate
\end{minipage} & \begin{minipage}[t]{0.21\columnwidth}\raggedright
Import approach
\end{minipage} & \begin{minipage}[t]{0.22\columnwidth}\raggedright
Data families
\end{minipage} & \begin{minipage}[t]{0.23\columnwidth}\raggedright
Type families (only stencil data type)
\end{minipage}
\\\noalign{\medskip}
\begin{minipage}[t]{0.14\columnwidth}\raggedright
Repetition viscosity
\end{minipage} & \begin{minipage}[t]{0.21\columnwidth}\raggedright
\textbf{Worst}

Clearly here we have a very high viscosity: each function must be re
written in terms of new syntax and run in different ways.
\end{minipage} & \begin{minipage}[t]{0.21\columnwidth}\raggedright
We have improved the viscosity significantly. The sure must only
implement their stencils in one language but they must still change all
the imports and correct type errors.
\end{minipage} & \begin{minipage}[t]{0.22\columnwidth}\raggedright
Data families worsen the viscosity over the import method as we must now
change all the data constructors as opposed to the imports. In real code
there will be more of these than import locations.
\end{minipage} & \begin{minipage}[t]{0.23\columnwidth}\raggedright
\textbf{Best}

Here we have the least repetition viscosity of all the approaches. We
now only need to change the quasi quoter to change the whole
implementation.
\end{minipage}
\\\noalign{\medskip}
\begin{minipage}[t]{0.14\columnwidth}\raggedright
Imposed lookahead
\end{minipage} & \begin{minipage}[t]{0.21\columnwidth}\raggedright
\textbf{Worst}

The user must know ahead of time that they will be writing in two
languages to be sure to minimize duplication of code and structure their
program correctly.
\end{minipage} & \begin{minipage}[t]{0.21\columnwidth}\raggedright
\textbf{Best}

There is practially no imposed lookahead as we can simple swap out the
implementation by importing form different places.
\end{minipage} & \begin{minipage}[t]{0.22\columnwidth}\raggedright
\textbf{Best}

We do not have imposed lookahead as we can easily swap the constructors.
\end{minipage} & \begin{minipage}[t]{0.23\columnwidth}\raggedright
\textbf{Best}

There is little imposed look ahead in theory though some operations are
currently not supported in the GPU implementation. (TODO: link to more)
Some types may not be supported easily in both implementations so this
should be considered too.
\end{minipage}
\\\noalign{\medskip}
\begin{minipage}[t]{0.14\columnwidth}\raggedright
Consistency
\end{minipage} & \begin{minipage}[t]{0.21\columnwidth}\raggedright
\textbf{Worst}

The syntax's are different and so fairly inconsistent. There are,
however, some similarities between the two in their stencil
representation.
\end{minipage} & \begin{minipage}[t]{0.21\columnwidth}\raggedright
Consistency is improved as the syntax is now uniform but types are not
uniform.
\end{minipage} & \begin{minipage}[t]{0.22\columnwidth}\raggedright
The syntax and usage is the same except for changing the constructors
which is inconsistent.
\end{minipage} & \begin{minipage}[t]{0.23\columnwidth}\raggedright
\textbf{Best}

We have eliminated the inconsistency in usage of the data families. Now
the approach is almost entirely consistent except for the types.
\end{minipage}
\\\noalign{\medskip}
\begin{minipage}[t]{0.14\columnwidth}\raggedright
Terseness
\end{minipage} & \begin{minipage}[t]{0.21\columnwidth}\raggedright
\textbf{Worst}

A lot of code is written by the user to cope with the two different
implementations.
\end{minipage} & \begin{minipage}[t]{0.21\columnwidth}\raggedright
Changing requires a fair bit of code to be changed as we may be
importing many different things from the Ypnos libraries and all these
things must be changed.
\end{minipage} & \begin{minipage}[t]{0.22\columnwidth}\raggedright
We require a lot of code to express the swap from CPU to GPU.
\end{minipage} & \begin{minipage}[t]{0.23\columnwidth}\raggedright
\textbf{Best}

The syntax for switching is minimally terse.
\end{minipage}
\\\noalign{\medskip}
\begin{minipage}[t]{0.14\columnwidth}\raggedright
Hidden dependencies
\end{minipage} & \begin{minipage}[t]{0.21\columnwidth}\raggedright
\textbf{Best}

No hidden dependencies. It is very clear and explicit what is going on.
\end{minipage} & \begin{minipage}[t]{0.21\columnwidth}\raggedright
\textbf{Worst}

Many hidden dependencies are introduced as the types of the different
imported functions don't necessarily match. This can cause failures in
many different places on changing the import.
\end{minipage} & \begin{minipage}[t]{0.22\columnwidth}\raggedright
\textbf{Best}

Here the dependencies introduced in the type system by the import
approach have been made explicit by data constructors.
\end{minipage} & \begin{minipage}[t]{0.23\columnwidth}\raggedright
\textbf{Worst}

More hidden dependencies are introduced. Types change underneath the
users noses due to different type and constraint families. This might
affect some programs.
\end{minipage}
\\\noalign{\medskip}
\begin{minipage}[t]{0.14\columnwidth}\raggedright
Abstraction gradient
\end{minipage} & \begin{minipage}[t]{0.21\columnwidth}\raggedright
\textbf{Best}

The abstraction is at it's basic level: that of using Ypnos or
Accelerate. The user must get to grips with these abstractions as a
minimum.
\end{minipage} & \begin{minipage}[t]{0.21\columnwidth}\raggedright
The abstraction level is fairly low as it uses only simple Haskell
constructs.
\end{minipage} & \begin{minipage}[t]{0.22\columnwidth}\raggedright
The user must now be familiar with the idea of an associated data family
and GADT which are quite advanced Haskell type features.
\end{minipage} & \begin{minipage}[t]{0.23\columnwidth}\raggedright
\textbf{Worst}

The abstraction here is perhaps highest of all as it uses the most
advanced type features.
\end{minipage}
\\\noalign{\medskip}
\begin{minipage}[t]{0.14\columnwidth}\raggedright
Closeness of mapping
\end{minipage} & \begin{minipage}[t]{0.21\columnwidth}\raggedright
\textbf{Best}

This way we preserve the comonadic nature of the operations in the type
so that it is obvious to the user what is going on.
\end{minipage} & \begin{minipage}[t]{0.21\columnwidth}\raggedright
\textbf{Worst}

The comonadicity is lost due to having to change the types to suit the
accelerate implementation.
\end{minipage} & \begin{minipage}[t]{0.22\columnwidth}\raggedright
\textbf{Worst}

The comonadicity is lost also.
\end{minipage} & \begin{minipage}[t]{0.23\columnwidth}\raggedright
\textbf{Worst}

The comonadicity is lost.
\end{minipage}
\\\noalign{\medskip}
\hline
\noalign{\medskip}
\caption{Comparison of the different API's using cognitive dimensions.}
\end{longtable}

\paragraph{Conclusion}

As we now can see, the best approach for our users is that of associated
type families with the data constructor for the stencil function. This
approach is best in the viscosity, imposed lookahead, consistency and
terseness dimensions. However, for this it has compromised in hidden
dependencies, abstraction and closeness of mapping.

The \emph{hidden dependency} problems are mitigated by the Haskell
compiler which warns and throws errors when there is a conflict in these
dependencies. While a little increase in hidden dependencies is
necessary to reduce viscosity, there could be room for improvement here
by making the types more consistent. This would help us remove the
dependencies due to the changing types and constraints.

Given that our example users are fairly advanced the increase in
\emph{abstraction} should not be a problem however we should be aware of
this extra difficulty to learning the language. We imagine that Kiaran
wouldn't have a problem learning about type families but it is still a
learning curve.

The \emph{closeness of mapping} is an issue that is not inherent in the
implementation but rather an artifact of it. With more time on this
project I would try to re-introduce the comonadic types to the type
family approach. This could require using a lower level implementation
rather than using Accelerate. For this reason getting a closer mapping
was beyond the scope of this project.

\section{Conclusion}
