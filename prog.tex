%% LyX 2.0.5 created this file.  For more info, see http://www.lyx.org/.
%% Do not edit unless you really know what you are doing.
\documentclass[british]{scrartcl}
\usepackage[T1]{fontenc}
\usepackage{geometry}
\geometry{verbose,tmargin=3cm,bmargin=3cm,lmargin=3cm,rmargin=3cm}
\setcounter{secnumdepth}{0}
\setcounter{tocdepth}{0}

\makeatletter
%%%%%%%%%%%%%%%%%%%%%%%%%%%%%% User specified LaTeX commands.
%% You can modify the fonts used in the document be using the
%% following macros. They take one parameter which is the font
%% changing command.
%% \headerfont: the font used in both headers.
%%              Defaults to sans serif.
%% \titlefont:  the font used for the title.
%%              Defaults to \LARGE sans-serif semi bold condensed.
%% \sectionfont: the font used by \section when beginning a new topic.
%%              Defaults to sans-serif semi bold condensed.
%% \itemfont:   the font used in descriptions of items.
%%              Defaults to sans-serif slanted.
% to make your name even bigger, uncomment the following line:
% \titlefont{\Huge}
%%
%% You can modify the following parameters using \renewcommand:
%% \topicmargin: the left margin inside topics.
%%               Defaults to 20% of the column width (0.20\columnwidth).
% To get more room for left column of Topic layouts, uncomment following line:

\makeatother

\usepackage{babel}
\begin{document}

\subtitle{GPU Acceleration of the Ypnos Programming Language}


\title{Progress Report}


\author{Sam Pattuzzi (sp598@cam.ac.uk)}

\maketitle
\noindent \textbf{Project Supervisor:} D. A. Orchard \vspace{0.2in}


\noindent \textbf{Director of Studies:} Dr A. R. Beresford \vspace{0.2in}
 

\noindent \textbf{Project Overseers:} Dr A. Madhavapeddy \& Dr M.
Kuhn

\vfill{}



\subsection*{Report}

At the end of the Christmas break the project was two weeks behind schedule: of
the two primitives that needed to be implemented by that point only one, the
``run'' primitive, had been which left the ``reduce'' primitive. However, work
proceeded according to the plan included below and I am now back on track.
Work on the evaluation is well underway and producing promising results.

Learning the Haskell language and getting familiar with the type system and
concepts took longer than previously anticipated. The Haskell type system is
highly complex with many ``extensions'' which are required for this project.
The project was brought back on track after discovering that it would no longer
be necessary to write a compiler from Haskell AST to Accelerate AST as they are
one and the same. This significantly reduced the menial work without reducing
the technical merit. However, while building the translation and primitives, I
discovered that the types of the stencils and primitives need to be changed in
order to be consistent with their unaccelerated counterparts. Reconciliation of
these types took further week not previously accounted for in the plan (see
revised plan). This leaves the evaluation to be completed before I can begin
work on the dissertation. 

So far, the following has been accomplished: learning the Haskell language;
writing sample applications in Ypnos and Accelerate to get familiar with those;
setting up a unit testing framework and testing the written code; translation
from Ypnos stencil functions to Haskell functions (which are used by
Accelerate); implementation of the ``run'' and ``reduce'' primitives; unit
tests for the previous; informally evaluating performance on the GPU.


\subsection*{Revised Plan}
\begin{itemize}
\item \textbf{7$^{th}$ of January -- 20$^{th}$ of January (Christmas)}
Finish the primitives if necessary. Write the progress report. Start
work on the basic test bench.
\item \textbf{21$^{th}$ of January -- 3$^{rd}$ of February} Finish the
``reduce'' primitive and write unit tests to cover it. Unify the
types such that both accelerated and unaccelerated operation appear
the same to the user.
\item \textbf{4$^{th}$ of February -- 17$^{th}$ of February} Finalise
the main test bench and run experiments. Analyse the performance and
scalability of the approach. Make improvements to the code as necessary
to achieve the main aim of the project.
\item \textbf{18$^{th}$ of February -- 3$^{rd}$ of March} Write the main
chapters of the dissertation.
\item \textbf{4$^{th}$ of March -- 15$^{th}$ of March} Elaborate on the
existing tests bench and run final experiments. Complete most of dissertation
in draft form.
\item \textbf{16$^{th}$ of March -- 22$^{nd}$ of April (Easter)} Time
to review and tweak the dissertation. The rest of the vacation is
set aside for course revision.
\item \textbf{23$^{th}$ of April -- 5$^{th}$ of May} A draft must be completed
and sent to my supervisor and director of studies by the \textbf{23$^{rd}$
of April}. This will be followed by proof reading and then an early
submission.\end{itemize}

\end{document}
